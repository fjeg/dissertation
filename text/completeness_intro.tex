Given that the final result of a mammogram is a decision whether to follow-up on patients with mammographic lesions, this decision should be the primary driver of determining whether enough information has been provided in the report. 
Early approaches to measuring whether medical diagnostic tests were necessary involved calculating thresholds for posterior probabilities that would warrant more testing or treatment \cite{Pauker:1980cg}. 
Such methods required that the practicing physician provide the posterior probability. 
The Pathfinder system used a value of information calculation to repeatedly request more information for diagnosis until there was only one possible diagnosis left \cite{Heckerman:1992uq}.
This does not provide a flexible enough framework for stopping in probabilistic decision support systems for mammography since multiple diagnoses can have non-zero probabilities even when all descriptors are observed.
The STOP criteria provides a quantitative algorithm for when to stop requesting information and make a decision, but this is formulated only to measure whether the probability of an event exceeds a certain threshold \cite{Gaag:2011gs}.
This application measures whether more tests are needed to make a positive diagnosis, but does not solve the issue of whether more information will improve diagnostic error.