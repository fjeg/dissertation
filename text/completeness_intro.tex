Missing descriptors in mammography reports do not necessarily mean that the report does not have all the information to make a correct and justified diagnosis. Conversely, there are cases when most of the descriptors are reported, but the report still reaches an ambiguous diagnosis. Completeness of reports needs to be sensitive to the context of the information already given as well as the effect of missing information on the diagnosis. 

Given that the final result of a mammogram is a decision whether to follow-up on patients with mammographic lesions, this decision should be the primary driver of determining whether enough information has been provided in the report. Early approaches to measuring whether medical diagnostic tests were necessary involved calculating thresholds for posterior probabilities that would warrant more testing or treatment \cite{Pauker:1980cg}. Such methods required that the practicing physician provide the posterior probability. The Pathfinder system used a value of information calculation to repeatedly request more information for diagnosis until there was only one possible diagnosis left \cite{Heckerman:1992uq}. This did not provide a flexible framework for stopping if there was more than one possible diagnosis outside of physician judgment. The STOP criteria provide a quantitative algorithm for when to stop requesting information and make a decision, but this is formulated only to measure whether the probability of an event exceeds a certain threshold \cite{Gaag:2011gs}.