I propose three major research opportunities that follow from this work to further improve performance in radiology. Incorporating natural language processing to interpret free-text reports, developing a clinical system for use in a reader study, and finally a fully-automated framework to perform end-to-end diagnosis.

\subsection{Structuring free-text reports for input to FASR}
The current model for FASR requires input from structured reporting systems that is already encoded in an easily computable format. Though there are commercial products to do this kind of reporting, the prevalent reporting model still generates free text reports. Previous work in our lab showed promise in developing natural language processing tools to extract structure from free-text reports \cite{Bozkurt:2014jw}, and we have begun studying how inputs derived from these tools affect the models in FASR. Namely, how much noise due to concept extraction can our models handle? If we can, in fact, structure free text report we can harness much larger data sets of radiology reports coupled with images.

\subsection{Integration into clinical practice}
I have developed the components for the FASR pipeline and evaluated them on retrospective data. The next logical step is a reader study to measure the impact such a system has on performance in practice. The ePad reporting environment has a plugin system to deploy arbitrary decision support tasks during the reporting process \cite{Rubin:2008uz}. Future researchers can leverage it to perform studies on radiologists in the Stanford hospitals and clinics.

\subsection{Automating image interpretation}
Radiology is unique in that all the information about a case is contained in computationally consumable formats. Images, medical records, and requisitions constitute most of the information involved in decisions. This provides a prime opportunity where combined computer aided diagnosis, interpretation, and reporting can form a fully automated computational radiologist pipeline. Using the annotative verification system, we can generate structured reports directly from images. Those structured reports can then be verified for completeness and consistency to create a diagnosis that virtually identical in form to a radiologists'. Given that volumes of images are growing drastically, offloading to an \emph{automatic triage reader} will allow radiologists to focus on difficult cases that require their expertise.