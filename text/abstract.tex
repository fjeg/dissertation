\prefacesection{Abstract}
Radiology is a powerful tool to detect and diagnose abnormalities by allowing doctors to visually inspect internal pathology that could not otherwise be seen. However, assessing radiological images is limited by variations among practitioners, including deficiencies in their reporting of these imaging examinations as well as in their interpretations. Three main sources of these variations in interpretation are incorrectness of observations in the images, incompleteness of the radiological observations reported to characterize the abnormalities, and inconsistency of these observations with respect to the radiologists' overall impression. I hypothesize that the quantification and enforcement of correctness, completeness, and consistency of radiological observations will improve the diagnostic accuracy and reduce variability of interpretation. To test this hypothesis, I formulate a decision support framework that provides feedback to radiologists during the reporting of their radiological observations. I develop this system by creating novel statistical models to link radiological observations, computational imaging features, and disease to recognize incorrectness, incompleteness and inconsistency in reporting. I then harness these models to create a quantifiable metric of observation quality. In this dissertation, I describe this system with the following specific aims: (1) developed methods to assess completeness and correctness of radiology reports, (2) evaluated these methods in two important radiological domains (mammography and liver CT), and (3) developed framework to provide feedback to radiologist to ensure consistency between report and diagnosis, improving diagnostic performance.