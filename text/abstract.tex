\prefacesection{Abstract}
Breast cancer affects 1 in 8 women in the United States. It is the second leading cause of cancer deaths amongst women, resulting in 39,000 deaths per year. As a result, much work has been done to optimize the evaluation of cancer through mammography. However, assessing mammograms is limited by variations among practitioners in practice, including deficiencies in their reporting of these imaging examinations as well as in their interpretations. Two main sources of these variations in practice are incompleteness of the radiological observations reported to characterize the abnormalities seen in images and inconsistency of these observations with respect to the radiologists' overall impression. We hypothesize that the quantification and enforcement of completeness and consistency of radiological observations will improve the positive predictive value of diagnostic mammography. We propose a decision support system that provides feedback to radiologists during the reporting of their radiological observations. We will develop this system by creating novel statistical models to link radiological observations, computational imaging features, and disease to recognize incompleteness and inconsistency in reporting. We will then harness these models to create a quantifiable metric of observation quality. We propose a research plan with the following specific aims: (1) To characterize breast lesions seen in mammography images by capturing computationally-derived (``quantitative'') imaging features and radiologist-derived observational (``semantic'') features, (2) develop a radiological Decision Support System (DSS), (3) evaluate our DSS in mammography practice. Our methods will lead not only to better diagnostic accuracy and positive predictive value of diagnostic mammography, but they will be extensible to other imaging domains and possibly to other medical domains where diagnostic reasoning is documented in dictated reports.

