MAMMOGRAPHY
Decision-support systems have been developed to improve upon mammography interpretation and diagnosis \cite{Garg:2005cb, Burnside:2000wl, ElizabethS:2005gc, Rubin:2005jg}, however most of these systems follow the Greek Oracle model of decision-support: they simply give an answer to the diagnostic task rather then assisting the radiologist to improve their own decision \cite{Miller:1990wg, Friedman:2009dx}. Additionally, such systems interrupt the traditional radiological workflow \cite{Morgan:2011ct}. We posit that improving the radiologist’s report during reporting time mitigates both of these issues and is the ideal time to deliver effective decision-support.

LIVER

Recent research has investigated computer-aided methods to support diagnosis by providing a database of annotated images that can be retrieved by similarity\cite{Napel:2010es} and further indicates that radiological observations drawn from a controlled vocabulary can lead to accurate diagnoses of liver tumor types from CT images\cite{Korenblum:2011gx}. These observations can be treated as features of the image derived from human semantic annotations. As a result, we refer to them as semantic features. These semantic features allow radiologists to make their observations consistent, explicit, and machine-accessible. Radiological tools such as the ePAD \cite{Rubin:2008uz} have been implemented with the purpose of recording these annotations in a facile manner.

Computational analysis of these images may overcome this issue by creating quantitative and unbiased descriptors of image features. These computational features can be computed directly from the image's pixel values, independent of the semantic features. Digital image processing techniques have been used to extract features indicative of lesion attenuation, texture \cite{Strela:1999ei, Zhao:2004gc}, edge and shape \cite{Hong:2006fl, Manay:2006cg, MRangayyan:2005td, Xu:2012bh}. We hypothesize that these computational features can be coupled with methods in statistical machine learning to form a framework to predict semantic features. This could be useful in building a decision support system to aid radiology interpretation.