In this study, we compare the performance of two widely used machine-learning methods to achieve this: logistic regression and L1-regularized logistic regression (LASSO). From these results, we can determine which semantic features are predicted by computational ones, and which computational features are the most useful for this purpose. Knowledge of which semantic features are not predicted well could lead to the development of new computational features for this purpose. Ultimately, prediction of semantic features from computational ones could lead to reduced variability in image interpretation.


The pixel data within the segmented liver lesions were processed to quantify contrast, texture, boundary, and shape (Table 2). Each lesion's extracted computational features were concatenated into a 431-dimensional feature vector.

Table 2: Computational features and their dimensions
Computational Feature Group	Dimension
Contrast	2
Proportion of pixels with intensity larger than 1100	1
Difference of means	1
Texture	349
Histogram	9
Histogram - Peak Position	1
Histogram - Entropy	1
Histogram - Haar	1
Histogram - Daubechies	324
Variance	1
Gabor	12
Edge	61
Edge Sharpness	60
Histogram on Edge	1
Shape	19
Compactness	1
Roughness	1
Local Area Integral Invariant	15
Radial Distance Signature	2
All Features	431


Contrast Features: 2-element feature vector containing: (a) the proportion of pixels with intensity larger than 1100 Hounsfield Units (HU) and (b) the difference in the mean intensity values for pixels inside the lesion and within a 5-pixel rim outside the liver lesion.

Texture Features: 349-element feature vector containing: (a) 13-element gray-level histogram-based, including the 9-bin histogram itself, the low frequency coefficients of its 3-level Haar wavelet transform, the abscissa of its peak, entropy, and its variance \cite{Strela:1999ei}, (b) 12-element Gabor features \cite{Zhao:2004gc} including the mean of the Gabor energy in the frequency domain over 3 scales and 4 orientations in a total of 12 bins, and (c) 324-element Daubechies features with the dominant sub-band in a 2-scale Daubechies wavelet transform \cite{Wang:1998jv}.

Margin Sharpness Features: 61-element feature vector computed as follows: (a) We recorded the image intensity values along normals to the lesion contour at multiple points and then fit a sigmoid function to these values.  Two parameters for the fitted sigmoid, scale and window, were used to characterize each line segment. The scale measures the difference in intensities outside and inside the lesion, and the window measures the width of the transition from the liver lesion to the surrounding normal liver at the boundary. Two 30-bin histograms for the scale and window parameters were then created to form a 60-element feature vector. (b) We also recorded the number of modes in the histogram of all pixels recorded from each normal \cite{Xu:2012bh}.

Shape Features: 19-element feature vector describing: (a) compactness  \cite{Duda:1973ul}, (b) roughness \cite{Kilday:1993jk}, (c) local area integral invariant descriptor including the mean and standard deviation for 5 different scales \cite{Hong:2006fl, Manay:2006cg}, (d) radial distance signatures including mean and standard deviation \cite{MRangayyan:2005td}.  


Classification
Logistic regression was used to calculate the probability of a computational feature vector representing a positive or negative semantic feature. In order to mitigate the potential for over-fitting due to a large number of predictors (431 computational features) compared to data points (79 lesions), we also used L1-regularization \cite{Tibshirani:1996wb} to weight features given sparsity in the computational feature set.
