\subsection{Mammography Screening and Diagnosis}
Breast cancer affects 1 in 8 women in the United States. It is the second leading cause of cancer deaths amongst women \cite{Siegel:2012kt}. Early detection has been shown to reduce the mortality of cancer by catching the disease while it is more easily treatable \cite{Baker:1982jg}. Mammography was developed in an effort to improve such early detection. 

Mammography is the use of x-ray imaging on the breast to identify any abnormalities. Radiologists presented with mammograms are tasked with two problems: detection and interpretation. Detection is the task of visually inspecting the mammogram and locating any possible abnormalities. Interpretation is evaluating whether detected abnormalities are suspicious for breast cancer.

Mammography has shown to be beneficial for early detection of breast cancer \cite{Nystrom:2002hb}. Currently, the American Cancer Society recommends that women with no specific risk for breast cancer get yearly screening mammograms to catch potentially malignant findings early \cite{Smith:2003en}.

\subsection{Interpration of Mammographic Findings}

Formally, the interpretation problem is defined as follows: A radiologist is presented with a lesion in a mammogram, patient history and demographics, and possibly prior mammograms. The radiologist must decide whether this lesion warrants no action or follow-up (either imaging or biopsy) based on their suspicion of malignancy. This suspicion of malignancy is quantified as the BI-RADS assessment category, which is an ordinal value ranging from 1 to 6. An additional assessment category of 0 is used to indicate there is not enough information in the mammogram to make a decision. These assessment categories were designed to have probabilistic interpretations, where each value has a range of posterior probabilities of malignancy as shown in Table \ref{table:birads}. A BI-RADS assessment of 1, 2, or 3 indicates the recommendation is no immediate follow-up (a negative assessment). A BI-RADS assessment of 4 or 5 indicates a recommendation for follow-up imaging or biopsy should be considered (a positive assessment). An assessment of 0 should not count as either positive or negative, but the fact that it necessitates immediate follow-up imaging means that it is treated as a positive finding \cite{Barlow:2004cy}. BI-RADS 6 is a non-diagnositc category used to indicate that the images reflect a known cancer diagnosis being evaluated for treatment planning. These assessment categories implicitly mean that any lesion with a posterior probability of greater than 2\% should be considered as a positive finding. Recent work has shown that this 2\% threshold rule is justified via epidemiological risk analysis \cite{Burnside:2012fk}. In addition to providing an assessment, radiologists must provide a report that justifies their decision. This report has a set of categorical descriptors standardized by BI-RADS, which can be interpreted as evidence for their decision.
\footnote{The description provided here is with relation to the 4th edition of BI-RADS, which was released in 2003. The 5th edition (released in 2013) of BI-RADS has altered the names of several descriptors and recommendations for assessment categories. Because all the mammography data used for analysis was collected before the release of the 5th edition, we will work under the framework of the 4th edition. All the methods described can easily be extended to adopt the conventions of the 5th edition.}

\begin{table}[h!]
\centering
\begin{tabular}{|c|c|c|}
	\hline  BI-RADS Assessment&  Probability of Malignancy & Description \\ 
	\hline\hline
	0& N/A & Additional Imaging Needed \\ 
	\hline
	1& 0\% & No Abnormality \\ 
	\hline  
	2& 0\% & Benign Finding  \\ 
	\hline  
	3& $<$ 2\% & Probably Benign Finding \\ 
	\hline  
	4& 2-95\% & Suspicious Abnormality \\ 
	\hline  
	5& $>$ 95\% & Highly Suggestive of Malignancy \\ 
	\hline  
	6& 100\% & Biopsy Proven \\ 
	\hline 
\end{tabular}
\caption{The BI-RADS assessment categories and their probabilistic interpretations.}
\label{table:birads}
\end{table}

\subsection{Mammography Screening: Controversy and Variability}
The American Cancer Society recommends annual screening mammography for women over 40 to detect breast cancer early when it is most treatable \cite{Nystrom:2002hb, Smith:2003en, Smart:1997hk}. However, this recommendation has been refuted by several longevity studies \cite{Bleyer:2012dc, Kalager:2012ez},  culminating in the United States Preventative Services Task Force recommending biennial screening after 50 years of age \cite{Kerlikowske:2013ej, Anonymous:2009fl}, which argues that unnecessarily early and frequent screening results in high economic and emotional costs.  False positives in screening drive much of these costs due to expensive follow-up treatment and long-term psychosocial harm to patients \cite{Kerlikowske:2013ej, Brodersen:2013kq}. While a reduction in screening is one possible solution to addressing the issue of false positive detections, it comes at the cost of possibly missing cancer at an early stage. An alternative solution is to directly improve radiologist performance by reducing their false positive interpretations of mammography. 

Like all screening tests, mammography trades off sensitivity with specificity, or equivalently, trading off false negative with false positive findings. These trade-offs are determined by a radiologists' personal model of practice, which is subjective. A conservative radiologist might practice at an increased sensitivity, meaning fewer false negative findings, at the cost of reduced specificity, meaning more false positives. Such subjectivity results in variability in mammography practice, which is a well-known and unsolved challenge \cite{Elmore:2009vu, Elmore:2012er, Beam:1996ui, Taplin:2008bv}. 

Though BI-RADS assessments have objective probabilistic underpinnings, mammography interpretation is inherently subjective. Modern practice traditionally does not include quantitative estimates of these probabilities. Rather, radiologists provide the assessment categories based on training and experience. The use of BI-RADS assessment categories allows us to evaluate radiological performance as if radiologists are binary classifiers. We can measure their true positives (TP), false positives (FP), true negatives (TN), and false negatives (FN) as well as all associated statistics (e.g. positive predictive value, sensitivity, specificity). Moreover, the use of categorical descriptors allows us to build joint models of their decision given evidence.

\subsection{Mammography decision support}
Decision-support in mammography is grouped into two classes, computer-aided detection (CADe) and computer-aided diagnosis (CADx). 

CADe is used to automatically identify regions of interest to guide the radiologist after they have initially viewed the image without help. The goal is to reduce false negatives in detection of malignant lesions. CADx aims to improve the interpretation of a finding after a radiologist has identified it. In mammography this traditionally means determining if a finding warrants follow-up in the form of additional imaging or biopsy. Currently, the FDA has cleared several CADe products for use in clinical practice but there has not been as much success for CADx outside of academic research \cite{Castellino:2005ke, Oliver:2010fm, Fujita:2008it}. Though CADe continues to present interesting research, this thesis will focus on CADx.

In computational parlance, CADx problems are classification problems to predict state of malignancy of a finding given its features. Features of findings can be derived directly from the mammogram via computer vision and image processing techniques \cite{Jiang:1999fj,Jiang:2001fy, Giger:2013jb, Eadie:2011cv} or they can be derived from human observations \cite{ElizabethS:2005gc, Burnside:2000wl, Rubin:2005jg}. We refer to these as computational features and semantic features, respectively. Computational features have the advantage that they can be quickly and reproducibly extracted from the image. Semantic features, due to their subjective human readers, are much more variable and noisy. Despite these shortcomings, semantic features present much higher level information about the image and can incorporate the wealth of prior knowledge by the readers \cite{Liberman:ws,Elter:2009fv}.

Studies into deployment of medical decision-support systems help us understand why CADx systems have not seen adoption in clinical practice. Most CADx systems follow the Greek Oracle model of decision-support: they simply give an answer to the diagnostic task rather then assisting the radiologist to improve their own decision \cite{Miller:1990wg, Friedman:2009dx, Bright:2012ga}. Additionally, such systems interrupt the traditional radiological workflow \cite{Morgan:2011ct}. Despite these shortcomings, computer-aided diagnosis (CADx) could potentially diminish subjectivity in the interpretation of clinical information using quantitative methods and objective decision points. In addition, it would be possible to directly measure and tune performance in CADx systems; a task that is much more challenging in unassisted human readers. The difference between such systems and radiologists is that this tradeoff in CADx systems is explicitly set as the “operating point” (a particular value of sensitivity and the consequent specificity) on the receiver operating characteristic (ROC) curve of the system, which is used to assess and maximize the performance of CADx systems. In probabilistic CADx systems, a probabilistic threshold solely determines the operating point. This threshold can be interpreted as the minimum probability of cancer that a lesion must exhibit before it is deemed a positive finding (i.e. considered for biopsy). Most radiologists strive to have a fixed operating point, but given the qualitative nature of mammography interpretation, it is not possible for them to know their probabilistic threshold precisely. This is problematic since BI-RADS states that a probability of malignancy greater than 2\% should be considered a positive result \cite{Liberman:ws,Liberman:2002gg}; however, at present, there is no way to measure what threshold a radiologist is using. This makes it difficult to quantify if individual radiologists are practicing too conservatively by setting their threshold low to avoid missing cancers. While it is of utmost importance to detect cancer, an unnecessarily low threshold means more false positive detections. Even if a radiologist were deemed to be practicing too conservatively, it would be difficult to tune performance to practice less aversely.














