Decision support for radiologists can come in several forms based on the decision being made. Prevalent decision support methods involves the synthesis and retrieval of large data sets for reference, computer-aided detection of findings in images (CADe), computer-aided diagnosis of findings in images (CADx), and computer-aided recommendations of clinical management. I provide a brief survey of some prominent methods and works here.


Several studies on decision support in radiology show that these systems often interrupt clinical  work-flow, give assessments rather than recommendations, and fail to provide decision-support during decision-making \cite{Kawamoto:2005gn,Morgan:2011ct}. In general, they still follow the the so-called \emph{Greek Oracle} model of decision support, where the radiologist simply inputs information to a system so that it may make a final decision \cite{Miller:1990wg,Miller:1994cx}. Such systems fail when deployed in practice since they are based on the implicit assumption that computers can perform the duties of doctors better than doctors themselves. Charles Friedman goes so far as to declare that the \emph{Fundamental Theorem of Biomedical Informatics} is that decision support should ``augment human reasoning'' beyond the capabilities of an unaided practitioner \cite{Friedman:2009dx}. Thus, it is crucial to develop radiological decision support tools that fit into the work-flow and augment their reasoning. A perfect window to provide this support is during radiological reporting.


\subsection{Radiology reference and data mining}
One of the earliest forms of decision support allows radiologists to query reference material and images. Generation of imaging databases that are searchable is non-trivial, involving significant time cleaning and annotating data. Natural language processing and computational image analysis allows practitioners to significantly speed up such work  \cite{Depeursinge:2012ce, Bozkurt:2014jw,Nassif:2009du}. In modern data-rich environments this has morphed into querying similar images to the finding in question via content-based image retrieval (CBIR) \cite{Akgul:2011ey}. CBIR frameworks require methods to represent data in a meaningful way and methods to compute the similarity between data points via these representations. Representations of images typically involve computational feature extract to quantify intensity, texture, edge, and shape \cite{Strela:2002vq,Zhao:2005wb, Hong:2006ti,Manay:2006un,MRangayyan:2005td,Xu:2012bh}, and semantic annotations of images via descriptors from radlex \cite{Langlotz:2006jn}. Similarity metrics can come in the form of basic vector distances to sophisticated classification frameworks \cite{Akgul:2011ey}. Modern systems like the biomedical imaging metadata manager (BIMM) makes use of computational and semantic descriptors in conjunction with a logistic regression framework for similarity in liver lesions \cite{Korenblum:2011gx}. More novel methods make use of automatically learned representations from data via deep learning \cite{Shin:2015wl}. 

\subsection{Computer-aided detection in images}
CADe is used to automatically identify regions of interest to guide the radiologist after they have initially viewed the image without help. The goal is to reduce false negatives in detection of relevant findings. CADe has also been the most commercially successful form of radiological decision support, with several products on the market and prominent usage amongst practicing radiologists \cite{Castellino:2005ke}. Generally, CADe systems perform computational image analysis on the medical image to extract features characterizing local regions in the image. These characterizations are used to create a saliency map of the entire image. Then techniques such as non-maximal suppression identify single points of high interest and deliver the marking to the radiologists. These markings are carefully tuned to have high degrees of sensitivity while balancing false positive marks \cite{Oliver:2010fm}. Mammography has seen the widest adoption of these methods for detection of calcifications and masses \cite{Cheng:2003ig,Castellino:2005ke,Meeuwis:2010bv,Oliver:2010fm,Fenton:2011fw,Fenton:2012kz,Jamieson:2012hz,Gallas:2012eg,Giger:2013jb}.


\subsection{Computer-aided diagnosis in images}

\cite{Jiang:1999fj,ElizabethS:2005gc,Gallas:2012eg,Bright:2012ga,Giger:2013jb,Depeursinge:2010jl,Fujita:2008it,Eadie:2011cv,Rubin:2005jg,Garg:2005cb,Elter:2009fv,Jamieson:2010vl,Jamieson:2010tt,Cheng:2003ig,Jiang:2001fy}
CADx problems are classification problems to predict state of malignancy of a finding given its features. Features of findings can be derived directly from the mammogram via computer vision and image processing techniques \cite{Jiang:1999fj,Jiang:2001fy, Giger:2013jb, Eadie:2011cv} or they can be derived from human observations \cite{ElizabethS:2005gc, Burnside:2000wl, Rubin:2005jg}. We refer to these as computational features and semantic features, respectively. Computational features have the advantage that they can be quickly and reproducibly extracted from the image. Semantic features, due to their subjective human readers, are much more variable and noisy. Despite these shortcomings, semantic features present much higher level information about the image and can incorporate the wealth of prior knowledge by the readers \cite{Liberman:ws,Elter:2009fv}.


Studies into deployment of medical decision-support systems help us understand why CADx systems have not seen adoption in clinical practice. Most CADx systems follow the Greek Oracle model of decision-support: they simply give an answer to the diagnostic task rather then assisting the radiologist to improve their own decision \cite{Miller:1990wg, Friedman:2009dx, Bright:2012ga}. Additionally, such systems interrupt the traditional radiological workflow \cite{Morgan:2011ct}. Despite these shortcomings, computer-aided diagnosis (CADx) could potentially diminish subjectivity in the interpretation of clinical information using quantitative methods and objective decision points. In addition, it would be possible to directly measure and tune performance in CADx systems; a task that is much more challenging in unassisted human readers. The difference between such systems and radiologists is that this tradeoff in CADx systems is explicitly set as the “operating point” (a particular value of sensitivity and the consequent specificity) on the receiver operating characteristic (ROC) curve of the system, which is used to assess and maximize the performance of CADx systems. In probabilistic CADx systems, a probabilistic threshold solely determines the operating point. This threshold can be interpreted as the minimum probability of cancer that a lesion must exhibit before it is deemed a positive finding (i.e. considered for biopsy). Most radiologists strive to have a fixed operating point, but given the qualitative nature of mammography interpretation, it is not possible for them to know their probabilistic threshold precisely. This is problematic since BI-RADS states that a probability of malignancy greater than 2\% should be considered a positive result \cite{Liberman:ws,Liberman:2002gg}; however, at present, there is no way to measure what threshold a radiologist is using. This makes it difficult to quantify if individual radiologists are practicing too conservatively by setting their threshold low to avoid missing cancers. While it is of utmost importance to detect cancer, an unnecessarily low threshold means more false positive detections. Even if a radiologist were deemed to be practicing too conservatively, it would be difficult to tune performance to practice less aversely.
\subsection{Computer-aided recommendations of clinical management}













