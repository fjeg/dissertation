Findings in radiology are communicated primarily via text-reports.


Typical radiology report creation is done via dictation while the radiologist is interpreting the image. This reporting process is unique in medicine since the report is created \emph{during} the workup and analysis of the patient \cite{Noumeir:2006cb}.


Considerable effort has been made to improve upon the variability in report content and diagnosis; arguably the most successful of which is in mammography \cite{Langlotz:2009fn,Burnside:2009ki}. The Breast Imaging-Reporting and Data System (BI-RADS) provides a standard lexicon of descriptors for radiological observations \cite{Liberman:ws}. This standardized language has virtually eliminated ambiguity in terminology across the United States. Further work has been done to create the RadLex ontology for a more universal solution to radiological terminology \cite{Langlotz:2006jn}. Yet, there is still a substantial body of work revealing variability with regard to sensitivity and specificity of evaluation of lesion malignancy amongst different readers as well as different facilities \cite{Jackson:2009fw, Beam:1996ui, Elmore:2002vc, Taplin:2008bv}. Such variability is not restricted to diagnosis and impressions; even the \emph{findings} within the contents of the report have shown variability \cite{Hobby:2000th, Robinson:1997uq}.