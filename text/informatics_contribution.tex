I have developed several informatics methods to measure and improve upon radiological practice These methods can be extended to medical domains both outside of mammography and as well as outside of radiology. 

\subsubsection{Computational prediction of radiologist annotations from medical images}
I have developed methods to predict the annotations a radiologist will make regarding an abnormality in an image solely from the image data. I have shown that this method can achieve high accuracy in such predictions and can be used to validate information in a radiological report. This method demonstrates the value of large feature vectors coupled with strong regularization to perform prediction in signal data such as images.

\subsubsection{A metric for completeness of report information}
I have developed a novel method to measure how complete the information is in a structured report that takes into account the nuance of the abnormality. This metric not only allows us to measure report quality, but it is also correlated with error in reporting. This method can be used in any generative predictive model to measure the information content of the evidence.

\subsubsection{A systematic method to deliver feedback to medical reports}
I propose a new framework for the development of delivering optimal feedback in decision-support systems. I do this by decoupling feedback criteria from stopping criteria and propose an evaluation framework that directly improves classification accuracy under data budget constraints. This can generalize to any evidence-gathering feedback model with an attached decision.