In this study I presented a framework for predicting radiological observations of liver lesions using computational image features. I computed a wide array of computational features from CT images of liver lesions and used these features to train logistic regression and LASSO classifiers. I was able to conclude that LASSO classification is well-suited for classification problems in the medical domain due to its ability to handle data sets that are small relative to their feature space.

\subsection{Regularization is necessary for medical image classification}
I experimented with other classifiers such as k-nearest-neighbor, support vector machines, and latent discriminant analysis. These methods were severely hampered by such high dimension/low sample sized data, resulting in slow (or no) model convergence as well as poor performance. In the interest of training time, I ended evaluation of these models as they showed no promise warranting further investigation. The logistic regression model family was both time efficient to train and had the most promising results. In addition, the ability to use the $L_1$ norm to regularize this problem was attractive given the conditions of this classification. As a result, this study focused on unregularized and regularized logistic regression. 

\subsection{Radiological descriptors require diverse feature sets}
The classification results provided interesting insights into tasks that computer vision can automate for humans. Namely, I was able to see what semantic features are difficult to characterize via computers by measuring how predictable they are. For example, two of the worst classification performances were predicting the shape features \emph{round} or \emph{ovoid}. This is despite the fact that the feature vectors characterizing these lesions had several values specifically designed to quantify shape. One possible reason for this discrepancy between computational and semantic features is that round and ovoid are inherently subjective terms. Hence, there might exist human variability in such descriptors that cannot be accounted for computationally. This lends credence to further use of computational features for characterizing lesion shape, as there is no variability in these methods. 

Another example the showed the opposite effect: high performance predicting semantic features that had no explicit quantification via the computational features. In this case, I could predict ``multiple lesions > 10'' with high accuracy even though the feature vector was only meant to characterize a single lesion. A possible explanation here is that the mechanism of disease that causes multiple lesions also has correlated morphological effects on single lesions. Thus, the model can indirectly predict number of lesions based on analysis of only one such lesion in the group.

\subsection{Automatic annotation to ensure correctness}
Our approach can be used to evaluate the predictive value of computational features as well as to determine radiological observations that are difficult to predict from computational image features. While computational analysis is not likely to replace the trained eyes of a radiologist, this work can be used to develop decision support tools to increase accuracy and efficiency of radiological diagnosis.




