\subsection{Decision support for liver lesions}
Recent research has investigated computer-aided methods to support diagnosis by providing a database of annotated images that can be retrieved by similarity\cite{Napel:2010es} and further indicates that radiological observations drawn from a controlled vocabulary can lead to accurate diagnoses of liver tumor types from CT images\cite{Korenblum:2011gx}. These observations can be treated as features of the image derived from human semantic annotations. As a result, we refer to them as semantic features. These semantic features allow radiologists to make their observations consistent, explicit, and machine-accessible. Radiological tools such as the ePAD \cite{Rubin:2008uz} have been implemented with the purpose of recording these annotations in a facile manner.

Computational analysis of these images may overcome this issue by creating quantitative and unbiased descriptors of image features. These computational features can be computed directly from the image's pixel values, independent of the semantic features. Digital image processing techniques have been used to extract features indicative of lesion attenuation, texture \cite{Strela:2002vq,Zhao:2005wb}, edge and shape \cite{Hong:2006ti,Manay:2006un,MRangayyan:2005td,Xu:2012bh}.


\subsection{Mammography decision support}
Decision-support in mammography is grouped into two classes, computer-aided detection (CADe) and computer-aided diagnosis (CADx). CADe is used to automatically identify regions of interest to guide the radiologist after they have initially viewed the image without help. The goal is to reduce false negatives in detection of malignant lesions. CADx aims to improve the interpretation of a finding after a radiologist has identified it. In mammography this traditionally means determining if a finding warrants follow-up in the form of additional imaging or biopsy. Currently, the FDA has cleared several CADe products for use in clinical practice but there has not been as much success for CADx outside of academic research \cite{Castellino:2005ke, Oliver:2010fm, Fujita:2008it}. Though CADe continues to present interesting research, this thesis will focus on CADx. 

CADx problems are classification problems to predict state of malignancy of a finding given its features. Features of findings can be derived directly from the mammogram via computer vision and image processing techniques \cite{Jiang:1999fj,Jiang:2001fy, Giger:2013jb, Eadie:2011cv} or they can be derived from human observations \cite{ElizabethS:2005gc, Burnside:2000wl, Rubin:2005jg}. We refer to these as computational features and semantic features, respectively. Computational features have the advantage that they can be quickly and reproducibly extracted from the image. Semantic features, due to their subjective human readers, are much more variable and noisy. Despite these shortcomings, semantic features present much higher level information about the image and can incorporate the wealth of prior knowledge by the readers \cite{Liberman:ws,Elter:2009fv}.

Studies into deployment of medical decision-support systems help us understand why CADx systems have not seen adoption in clinical practice. Most CADx systems follow the Greek Oracle model of decision-support: they simply give an answer to the diagnostic task rather then assisting the radiologist to improve their own decision \cite{Miller:1990wg, Friedman:2009dx, Bright:2012ga}. Additionally, such systems interrupt the traditional radiological workflow \cite{Morgan:2011ct}. Despite these shortcomings, computer-aided diagnosis (CADx) could potentially diminish subjectivity in the interpretation of clinical information using quantitative methods and objective decision points. In addition, it would be possible to directly measure and tune performance in CADx systems; a task that is much more challenging in unassisted human readers. The difference between such systems and radiologists is that this tradeoff in CADx systems is explicitly set as the “operating point” (a particular value of sensitivity and the consequent specificity) on the receiver operating characteristic (ROC) curve of the system, which is used to assess and maximize the performance of CADx systems. In probabilistic CADx systems, a probabilistic threshold solely determines the operating point. This threshold can be interpreted as the minimum probability of cancer that a lesion must exhibit before it is deemed a positive finding (i.e. considered for biopsy). Most radiologists strive to have a fixed operating point, but given the qualitative nature of mammography interpretation, it is not possible for them to know their probabilistic threshold precisely. This is problematic since BI-RADS states that a probability of malignancy greater than 2\% should be considered a positive result \cite{Liberman:ws,Liberman:2002gg}; however, at present, there is no way to measure what threshold a radiologist is using. This makes it difficult to quantify if individual radiologists are practicing too conservatively by setting their threshold low to avoid missing cancers. While it is of utmost importance to detect cancer, an unnecessarily low threshold means more false positive detections. Even if a radiologist were deemed to be practicing too conservatively, it would be difficult to tune performance to practice less aversely.








