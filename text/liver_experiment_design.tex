The purpose of this experiment was to evaluate the efficacy of $L_1$ regularization in the domain of medical images. I evaluated this by comparing Lasso classification to ordinary logistic regression.

\subsection{Liver data set}
With IRB approval, I obtained 79 de-identified CT images of liver lesions in the portal venous phase, including eight types of lesion diagnoses: metastasis, hemangioma (abnormal buildup of blood vessels in the liver), hepatocellular carcinoma, focal nodular hyperplasia (unknown cause), abscess (inflammation), laceration (injury or tear), fat deposition and cyst. For the initial development of image processing algorithms, I chose to focus on the most commonly used phase of imaging, the portal venous phase. For each scan, the axial slice with the largest lesion area was selected for analysis. A radiologist drew and recorded a Region of Interest (ROI) around the lesion on these images using the freely available OsiriX workstation \cite{Armato:2007ks,Rosset:2004kk}.

\subsection{Classifier performance}
I simultaneously performed model selection and assessment on the liver data set using nested leave-one-out cross-validation (LOOCV). The inner-loop of LOOCV acted as a validation set to perform model selection via the regularization parameter $\lambda$ and the classification threshold $t$. The outer-loop of LOOCV acted as a test set to to perform model assessment via the misclassification rate (MCR) and the area under the receiver operating characteristic curve (AUC).

\subsection{Statistical analysis}
I evaluated performance in predicting each of the 30 semantic features via their AUC and MCR. I then used a paired two-sample t-test to compare these scores for significance. 

\subsection{Implementation}
All programming and statistical analysis was performed in MATLAB R2007a. LASSO classification was done using the glmnet package for MATLAB \cite{Friedman:2009wm}.