\subsection{Liver data set}
With IRB approval, I obtained 79 de-identified CT images of liver lesions in the portal venous phase, including eight types of lesion diagnoses: metastasis, hemangioma (abnormal buildup of blood vessels in the liver), hepatocellular carcinoma, focal nodular hyperplasia (unknown cause), abscess (inflammation), laceration (injury or tear), fat deposition and cyst. For the initial development of image processing algorithms, I chose to focus on the most commonly used phase of imaging, the portal venous phase. For each scan, the axial slice with the largest lesion area was selected for analysis. A radiologist drew and recorded a Region of Interest (ROI) around the lesion on these images using the freely available OsiriX workstation \cite{Armato:2007ks,Rosset:2004kk}.


\subsection{Semantic Features}
A radiologist annotated the ROI with the Electronic Physician's Annotation Device (ePAD; formerly iPAD) \cite{Napel:2010vb, Rubin:2008uz}. The ePAD system is based on a controlled radiological vocabulary called RadLex to define semantic features \cite{Korenblum:2011gx, Langlotz:2006jn}, and enforces complete description of the required aspects of the visualized lesion. We extended the RadLex terminology to include a broader array of descriptive terms for this study that more comprehensively describe liver lesions. Each annotation resulted in the creation of a binary semantic feature vector of length 76 to indicate positive or negative observations.
Semantic features were not all equally likely and ranged widely in annotation frequency. The entropy of each semantic feature was measured using the binary entropy function in order to determine the distribution of positive versus negative observations. The binary entropy function is defined as:

\begin{align}
H(X) = -p\log_2(p) - (1-p)\log_2(1-p)
\end{align}

Where:

\begin{align}
p = \frac{\textrm{positive observations}}{\textrm{total observations}}
\end{align}

$H(X)$ ranges from 0 to 1, achieving a maximum when a semantic feature has an equal number of positive and negative observations~\cite{MacKay:2003wc}. To avoid degenerate classification cases, only features with entropy greater than 0.4 were considered for classification, creating a total 30 semantic features per lesion (Table \ref{table:semanticfeaturelist}).

\begin{table}[ht!]
	\centering
	\begin{tabular}{|l|c|}
		\hline
		Semantic Feature & Entropy of Observations \\ \hline \hline
		internal nodules & 0.5116 \\ \hline
		lobulated margin & 0.5116 \\ \hline
		heterogeneous enhancement & 0.548 \\ \hline
		irregular margin & 0.548 \\ \hline
		irregularly shaped & 0.548 \\ \hline
		hypervascular & 0.5822 \\ \hline
		water density & 0.5822 \\ \hline
		abuts capsule of liver & 0.6451 \\ \hline
		poorly-defined margin & 0.6451 \\ \hline
		soft tissue density & 0.6451 \\ \hline
		centripetal fill-in & 0.6739 \\ \hline
		homogeneous retention & 0.6739 \\ \hline
		multiple lesions $>$10 & 0.6739 \\ \hline
		peripheral discontinuous nodular enhancement & 0.6739 \\ \hline
		multiple lesions 6-10 & 0.7012 \\ \hline
		homogeneous fade & 0.727 \\ \hline
		lobular & 0.8163 \\ \hline
		multiple lesions 1-5 & 0.8163 \\ \hline
		round & 0.8163 \\ \hline
		circumscribed margin & 0.8702 \\ \hline
		hypodense & 0.8702 \\ \hline
		homogeneous enhancement & 0.8859 \\ \hline
		nonenhancing & 0.914 \\ \hline
		solitary lesion & 0.9484 \\ \hline
		enhancing & 0.9738 \\ \hline
		heterogeneous & 0.9738 \\ \hline
		homogeneous & 0.9804 \\ \hline
		normal perilesional tissue & 0.9804 \\ \hline
		ovoid & 0.9971 \\ \hline
		smooth margin & 0.9971 \\ \hline
	\end{tabular}
	\caption{Semantic features with entropy above 0.4}
	\label{table:semanticfeaturelist}
\end{table}



\subsection{Computational Features}
The pixel data within the segmented liver lesions were processed to quantify contrast, texture, boundary, and shape (Table \ref{table:compfeaturelist}). The computational features to represent these concepts were re-purposed from previous work on image-centric content-based image retrieval\cite{Xu:2012bh,Napel:2010vb} which had been optimized for retrieval of semantic terms. Each lesion's extracted computational features were concatenated into a 431-dimensional feature vector .

\subsubsection{Contrast Features}
2-element feature vector containing: (a) the proportion of pixels with intensity larger than 100 Hounsfield Units (HU) and (b) the difference in the mean intensity values for pixels inside the lesion and within a 5-pixel rim outside the liver lesion.

\subsubsection{Texture Features}
We extracted a 349-element feature vector containing: (a) 13-element gray-level histogram-based, including the 9-bin histogram itself, the low frequency coefficients of its 3-level Haar wavelet transform, the abscissa of its peak, entropy, and its variance \cite{Strela:2002vq}, (b) 12-element Gabor features \cite{Zhao:2005wb} including the mean of the Gabor energy in the frequency domain over 3 scales and 4 orientations in a total of 12 bins, and (c) 324-element Daubechies features with the dominant sub-band in a 2-scale Daubechies wavelet transform.

\subsubsection{Margin Sharpness Features}
We extracted a 61-element feature vector computed as follows: (a) We recorded the image intensity values along normals to the lesion contour at multiple points and then fit a sigmoid function to these values.  Two parameters for the fitted sigmoid, scale and window, were used to characterize each line segment. The scale measures the difference in intensities outside and inside the lesion, and the window measures the transition from the liver lesion to the surrounding normal liver at the boundary. Two 30-bin histograms for the scale and window parameters were then created to form a 60-element feature vector. (b) We also recorded the number of modes in the histogram of all pixels recorded from each normal.

\subsubsection{Shape Features}
We extracted a 19-element feature vector containing: (a) Compactness \cite{Duda:1973ul}. (b) Roughness \cite{Kilday:1993jk}. (c) Local area integral invariant descriptor including the mean and standard deviation for 5 different scales~\cite{Hong:2006ti,Manay:2006un}. (d) Radial distance signatures including mean and standard deviation~\cite{MRangayyan:2005td}.

\begin{table}
	\centering
	\begin{tabular}{|l|c|}
		\hline
		Computational Feature Group Description & Dimension \\ \hline \hline
		\textbf{Contrast} & \textbf{2} \\ \hline
		\hspace{2pt} Proportion of pixels with intensity larger than 1100 & 1 \\ \hline
		\hspace{2pt} Difference of means & 1 \\ \hline
		\textbf{Texture} & \textbf{349} \\ \hline
		\hspace{2pt} Histogram & 9 \\ \hline
		\hspace{2pt} Histogram - Peak Position & 1 \\ \hline
		\hspace{2pt} Histogram - Entropy & 1 \\ \hline
		\hspace{2pt} Histogram - Haar & 1 \\ \hline
		\hspace{2pt} Histogram - Daubechies & 324 \\ \hline
		\hspace{2pt} Variance & 1 \\ \hline
		\hspace{2pt} Gabor & 12 \\ \hline
		\textbf{Edge} & \textbf{61} \\ \hline
		\hspace{2pt} Edge Sharpness & 60 \\ \hline
		\hspace{2pt} Histogram on Edge & 1 \\ \hline
		\textbf{Shape} & \textbf{19} \\ \hline
		\hspace{2pt} Compactness & 1 \\ \hline
		\hspace{2pt} Roughness & 1\\ \hline
		\hspace{2pt} Local Area Integral Invariant & 15\\ \hline
		\hspace{2pt} Radial Distance Signature & 2\\ \hline \hline
		\textbf{All Features} & \textbf{431} \\ \hline
	\end{tabular}
	\caption{Computationally-derived features and their dimensions}
	\label{table:compfeaturelist}
\end{table}

\subsection{Classification methods}
The classification task here is to predict each semantic feature (Table \ref{table:semanticfeaturelist}) using the computational features (Table \ref{table:compfeaturelist}).
There are a robust set of tools to perform this binary classification such as support-vector machines, Gaussian discriminant analysis, na\"{i}ve bayes, k-nearest-neighbor, and logistic regression. These classifiers all have their pros and cons, but none perform particularly well in the this domain because there are generally much fewer cases than features. The cost of obtaining more annotated images is significantly higher than extracting more meaningful computational features. As a result, this is a classic case of p >> n, or an overdetermined classification problem. This ill-conditioned problem allows any model to fit the data perfectly without improving generalization accuracy; a problem known as overfitting \cite{Hastie:2009wp,Friedman:2009wm}. We can overcome overfitting through regularization which reduces model complexity to improve generalization error. 

In this study, I compare the misclassification rate and area under the ROC curve of two widely used machine-learning methods for classification: logistic regression and $L_1$-regularized logistic regression. The $L_1$ regularization induces sparsity in the model weights effectively performing feature selection of computational features to predict semantic features. From these results, I can determine which semantic features are well-predicted and which computational features are the most useful for this purpose. Ultimately, prediction of semantic features from computational ones could lead to reduced variability in image interpretation.

\subsection{Statistical analysis}
\subsubsection{Classifier evaluation}
I simultaneously performed model selection and assessment on the liver data set using nested leave-one-out cross-validation (LOOCV). The inner-loop of LOOCV acted as a validation set to optimize misclassification rate by tuning the regularization parameter $\lambda$ and the classification threshold $t$. The outer-loop of LOOCV acted as a test set to perform model assessment via the misclassification rate (MCR) and the area under the receiver operating characteristic curve (AUC). I then used a paired two-sample t-test to compare these scores for significance. 

\subsubsection{Implementation}
All programming and statistical analysis was performed in MATLAB R2007a. LASSO classification was done using the glmnet package for MATLAB \cite{Friedman:2009wm}.