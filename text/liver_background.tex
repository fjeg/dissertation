\subsection{Computed Tomography of the Liver}
Liver lesions stem from a variety of causes and manifest in several variations on CT images; some are benign while others may be malignant, and the different diagnoses demonstrate a variety of visual appearances. The ability to differentiate these lesions efficiently and accurately is important to patient treatment and outcome. Contrast-enhanced CT imaging is the dominant technology used for liver lesion diagnosis \cite{Baron:1994vg}. This modality takes advantage of the fact that the liver receives blood from two main sources, the portal vein and the hepatic artery. The portal vein supplies about 80\% of blood to the liver with the hepatic artery providing the other 20\%. Due to varying physiology among liver lesions, differing lesion types may not share the same blood intake proportions as the surrounding liver tissue. Multi-phasic contrast-enhanced imaging takes advantage of this by obtaining images of the liver at multiple time points after injection of contrast agent. This allows for visualization of lesions due to the difference in time between the arrival of contrast agent in the hepatic and portal circulations. This difference causes several distinctive imaging features on contrast-enhanced imaging. As an example, primary liver cancer tumors receive all blood from the hepatic artery, so they contain higher concentrations of contrast agent than surrounding liver parenchyma during the arterial phase \cite{Lautt:1987wma,Matsui:1991vba}. Arterial-phase contrast-enhanced CT, therefore, may be helpful in finding masses that exhibit malignant tumor characteristics. Other phases may be useful for differentiating tumor types. For example, metastatic tumors appear less dense compared to the normal liver during the portal venous phase.

The difference in density between a lesion and its surrounding tissue at various times after the injection of iodine in a peripheral vein in multi-phasic imaging is called the temporal enhancement pattern. Analysis of a lesion's temporal enhancement pattern through the different phases of image acquisition helps radiologists to make diagnoses. Unfortunately, the specificity of this method is a function of the size of the lesion and prone to a high false positive rate because several types of liver lesions, including benign ones, have similar manifestations on CT images \cite{Lencioni:2005ia}. Human variability has also been shown to be a challenge for lesion differential diagnosis, and automated methods are being investigated to improve diagnosis of these lesions \cite{Armato:2007ks}. 

\subsection{Decision support for liver lesions}
Recent research has investigated computer-aided methods to support diagnosis by providing a database of annotated images that can be retrieved by similarity\cite{Napel:2010es} and further indicates that radiological observations drawn from a controlled vocabulary can lead to accurate diagnoses of liver tumor types from CT images\cite{Korenblum:2011gx}. These observations can be treated as features of the image derived from human semantic annotations. As a result, we refer to them as semantic features. These semantic features allow radiologists to make their observations consistent, explicit, and machine-accessible. Radiological tools such as the ePAD \cite{Rubin:2008uz} have been implemented with the purpose of recording these annotations in a facile manner.

Computational analysis of these images may overcome this issue by creating quantitative and unbiased descriptors of image features. These computational features can be computed directly from the image's pixel values, independent of the semantic features. Digital image processing techniques have been used to extract features indicative of lesion attenuation, texture \cite{Strela:2002vq,Zhao:2005wb}, edge and shape \cite{Hong:2006ti,Manay:2006un,MRangayyan:2005td,Xu:2012bh}.